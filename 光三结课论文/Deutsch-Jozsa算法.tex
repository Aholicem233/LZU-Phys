\subsection{Deutsch-Jozsa算法}
由David Deutsch和Richard Jozsa在1992年提出。该算法旨在解决所谓的Deutsch-Jozsa问题,即确定一个给定的二进制函数是常数函数还是均匀(balanced)函数。在经典计算中,确定函数是否为常数或平衡需要至少$2^{n-1}+1$次查询,其中N是函数可能输入的数量。然而,Deutsch-Jozsa算法只需一次函数查询就能解决这个问题,比任何经典算法都要快得多。
(常函数: 所有的$f(x)=0$或者1;
均匀(balanced)函数: 恰好一半$x$得$f(x) = 0$, 另一半$x$得$f(x) = 1$。)
\subsubsection{Deutsch-Jozsa算法描述}
Deutsch-Jozsa算法的量子线路如下:
\begin{define}{Deutsch-Jozsa算法量子线路}
    \begin{center}
    \begin{quantikz}[row sep=0.5cm, column sep=0.6cm]
        \lstick{$\ket{0}^{\otimes n}$} & \qw & \gate{H^{\otimes n}} & \qw & 
        \gate[2][2.0cm]{U_f}\gateinput[1]{$x$}\gateoutput[1]{$x$} & \qw & \gate{H^{\otimes n}} & \meter{} \\
        \lstick{$\ket{0}$} & \gate{X} & \gate{H} & \qw & \gateinput{$y$}\gateoutput{$y\oplus f(x)$} & \qw & \qw & \qw
    \end{quantikz}
    \end{center} 
\end{define}

$U_f$称为量子黑箱(Oracle),定义为:
$$
    U_f:\ket{x,y}\rightarrow\ket{x,y\oplus f(x)}
$$

若$|y\rangle = \frac{|0\rangle - |1\rangle}{\sqrt{2}}$,则有
$$
\begin{aligned}
    U_f : &|x\rangle \left( \frac{|0\rangle - |1\rangle}{\sqrt{2}} \right) \mapsto \left( \frac{U_f |x\rangle |0\rangle - U_f |x\rangle |1\rangle}{\sqrt{2}} \right)\\
    &= |x\rangle \left| 0 \oplus f(x) \right\rangle - |x\rangle \left| 1 \oplus f(x) \right\rangle\\
    &= |x\rangle \left( \frac{|0 \oplus f(x)\rangle - |1 \oplus f(x)\rangle}{\sqrt{2}} \right)
\end{aligned}
$$

当 $f(x) =$ 0 或 1 时,有
\[
\begin{cases} 
    f(x) = 0: & \frac{|0 \oplus f(x)\rangle - |1 \oplus f(x)\rangle}{\sqrt{2}} = \frac{|0\rangle - |1\rangle}{\sqrt{2}} \\ 
    f(x) = 1: & \frac{|0 \oplus f(x)\rangle - |1 \oplus f(x)\rangle}{\sqrt{2}} = \frac{|1\rangle - |0\rangle}{\sqrt{2}} = -\frac{|0\rangle - |1\rangle}{\sqrt{2}} 
\end{cases} 
\]

\[ \frac{|0 \oplus f(x)\rangle - |1 \oplus f(x)\rangle}{\sqrt{2}} = (-1)^{f(x)} \left( \frac{|0\rangle - |1\rangle}{\sqrt{2}} \right) \]

可见,$U_f$作用于 $|x\rangle \left( \frac{|0\rangle - |1\rangle}{\sqrt{2}} \right)$相当于从整体上添加了一个相位因子 $(-1)^{f(x)}$。

\[ U_f : |x\rangle \left( \frac{|0\rangle - |1\rangle}{\sqrt{2}} \right) \mapsto (-1)^{f(x)} |x\rangle \left( \frac{|0\rangle - |1\rangle}{\sqrt{2}} \right) \]

Deutsch-Jozsa算法工作过程如下:

1. 输入态: $|0\rangle^{\otimes n} \otimes |1\rangle$

2. $H^{\otimes n+1}$变换:

\[
|\psi_1\rangle = \left( \frac{|0\rangle + |1\rangle}{\sqrt{2}} \right)^{\otimes n} \otimes \frac{|0\rangle - |1\rangle}{\sqrt{2}} = \sum_{x \in \{0,1\}^n} \frac{|x\rangle\rangle}{\sqrt{2^n}} \otimes \frac{|0\rangle - |1\rangle}{\sqrt{2}}
\]

3. $\hat{U}$变换:

\[
|\psi_2\rangle = \hat{U}|\psi_1\rangle = \sum_{x \in \{0,1\}^n} \frac{(-1)^{f(x)}|x\rangle\rangle}{\sqrt{2^n}} \otimes \frac{|0\rangle - |1\rangle}{\sqrt{2}}
\]

4. $H^{\otimes n}$变换:

由 $H|y\rangle = \sum_{z=0,1} \frac{(-1)^{yz}}{\sqrt{2}} |z\rangle (y = 0 \text{和} 1)$ 得
\[
|\psi_3\rangle = \hat{U}|\psi_1\rangle = \sum_{x,z \in \{0,1\}^n} \frac{(-1)^{f(x) + x \cdot z}|z\rangle}{2^n} \otimes \frac{|0\rangle - |1\rangle}{\sqrt{2}}
\]

其中 $x \cdot z = \sum_{i=1}^{n} x_i z_i \mod 2$

5. 测量:

若 $f(x)$ 为常函数,则 $|z\rangle = |0\rangle^{\otimes n}$ 的系数为 $\sum_{x \in \{0,1\}^n} \frac{(-1)^{f(x)}}{2^n} = \pm 1$,此时 $|\psi_3\rangle = |0\rangle^{\otimes n} \otimes \frac{|0\rangle - |1\rangle}{\sqrt{2}}$,所以测量结果为所有量子位均为0。

若 $f(x)$ 为均匀函数,则 $|z\rangle = |0\rangle^{\otimes n}$ 的系数为零,所以测量结果不可能所有的量子位均为0。

\subsubsection{Deutsch-Jozsa算法Qiskit实现}
\bold{1.实现黑盒函数}

定义函数 \texttt{dj\_oracle(case, n)} 实现 Oracle,并将其构建为一个名为 Oracle 的自定义门,返回参数为新构建的自定义门的标识符。参数 $n$ 为输入量子比特的数目。参数 \texttt{case} 为字符串类型:当 \texttt{case} 为 \texttt{"constant"} 时,实现了常值函数;当 \texttt{case} 为 \texttt{"balanced"} 时,实现了平衡函数。
\begin{py}
\begin{lstlisting}
## oracle函数
def dj_oracle(case, n):
    oracle_qc = QuantumCircuit(n+1)
    # 平衡函数
    if case == "balanced":
        # [1,2**n)间的随机整数 b
        b = np.random.randint(1, 2**n)
        # 得到 b 对应的二进制字符串
        b_str = format(b, '0'+str(n)+'b')
        # 若b_str[qubit]为'1',则在q[qubit]上添加一个X门
        for qubit in range(len(b_str)):
            if b_str[qubit] == '1':
                oracle_qc.x(qubit)
        # 每个量子比特上添加 CNOT 门,q[n]为目标量子比特
        for qubit in range(n):
            oracle_qc.cx(qubit, n)
        # 若b_str[qubit]为'1',则在q[qubit]上添加一个X门
        for qubit in range(len(b_str)):
            if b_str[qubit] == '1':
                oracle_qc.x(qubit)

    # 常值函数
    if case == "constant":
        # 取随机整数0或1, 若为1,则在q[n]上添加一个X门
        output = np.random.randint(2)
        if output == 1:
            oracle_qc.x(n)

    oracle_gate = oracle_qc.to_gate()
    # 将量子线路 oracle_qc 封装为自定义门
    oracle_gate.name = "Oracle"
    return oracle_gate
\end{lstlisting}
\end{py}

例如,当$n=3$,$b=3$时利用以下代码可以得到平衡函数对应的量子线路:
\begin{py}
\begin{lstlisting}
from qiskit import QuantumCircuit 
import numpy as np
n = 3
oracle_qc = QuantumCircuit(n+1)
b = 3
b_str = format(b, '0'+str(n)+'b')
for qubit in range(len(b_str)):
    if b_str[qubit] == '1':
        oracle_qc.x(qubit)
for qubit in range(n):
    oracle_qc.cx(qubit, n)
for qubit in range(len(b_str)):
     if b_str[qubit] == '1':
        oracle_qc.x(qubit)
         
print(b_str)
oracle_qc.draw(output='mpl',filename='qc2.png')#获得png图片
\end{lstlisting}
\end{py}
\fig{平衡函数量子线路}{0.6}{code/qc2.png}
$q_0$,$q_1$ 和 $q_2$ 为输入量子比特,$q_3$ 为辅助量子比特。易知,输入为000、011、101、110时$q_3$进行了两次“非”,即$q_3$保持不变,对应$f(x)$为0。输入为111、001、100、010时$q_3$进行了一次或三次“非”,对应$f(x)$为1。故$f(x)$为平衡函数。

\bold{2.实现Deutsch-Jozsa算法的量子线路}
\begin{py}
\begin{lstlisting}
def dj_algorithm(oracle, n):
    dj_circuit = QuantumCircuit(n+1, n)
    # 设置输出量子比特
    dj_circuit.x(n)
    dj_circuit.h(n)
    # 设置输入寄存器
    for qubit in range(n):
        dj_circuit.h(qubit)
    # 添加 Oracle 门
    dj_circuit.append(oracle, range(n+1))
    # 输入寄存器各量子比特上添加 H 门
    for qubit in range(n):
        dj_circuit.h(qubit)
    # 测量输入寄存器
    for i in range(n):
        dj_circuit.measure(i, i)
    return dj_circuit



# 库函数输入
import numpy as np
from qiskit_aer import Aer
from qiskit import QuantumCircuit, transpile
from qiskit.visualization import plot_histogram

#实现4量子比特 Deutsch-Jozsa 算法量子线路
n = 4
oracle_gate = dj_oracle('constant', n)
# 平衡函数时,'constant'改为 'balanced'

dj_circuit = dj_algorithm(oracle_gate, n)
#dj_circuit.draw(output='mpl',filename='qc3.png')#获得png图片
\end{lstlisting}
\end{py}
\fig{4量子比特 Deutsch-Jozsa 算法量子线路图}{0.6}{code/qc3.png}

\bold{3.模拟运行Deutsch-Jozsa算法}
\begin{py}
\begin{lstlisting}
# 选择模拟器
simu = Aer.get_backend('aer_simulator')

# 编译运行电路
transpiled_dj_circuit = transpile(dj_circuit, simu)
results = simu.run(transpiled_dj_circuit).result()

# 获取测量结果
answer = results.get_counts(dj_circuit)
plot_histogram(answer)
\end{lstlisting}
\end{py}
测试结果测得 $|0000\rangle$ 的概率为 100\%对应 $f(x)$ 是常值函数。
测试结果测得 $|1111\rangle$ 的概率为 100\%对应 $f(x)$ 是平衡函数。(由于代码实现平衡函数时对其进行了限制,其他的态没有出现。)
\fig{常函数Deutsch-Jozsa 算法结果}{0.8}{code/const.png}
\fig{平衡函数Deutsch-Jozsa 算法结果}{0.8}{code/balanced.png}