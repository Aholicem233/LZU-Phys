\section{量子计算物理实现}
\subsection{离子阱量子计算}
离子阱量子计算是一种利用囚禁离子作为量子比特(qubit)进行量子信息处理的技术。它是量子计算领域中最具前景的方法之一,具有高保真度、长相干时间和精确控制等优势。
\subsubsection{量子位的物理实现}
在基于离子阱的量子计算中,存在两种不同类型的量子位实现方式:(1.)基态量子位,其中信息被编码在基态的两个超精细或塞曼子能级中,以及(2.)光学量子位,其中信息被编码在基态和光学可及的亚稳态激发态中。这两种类型的量子位需要不同的实验技术,特别是基态量子位通过双光子拉曼跃迁或直接微波激发来操纵。相比之下,对光学量子位的操作是通过激光器提供的共振光场来执行的。测量寄存器中量子位的状态通常通过电子搁置方法进行,对两种类型的量子位都使用辅助的短寿命状态。

为了实现量子算法,量子位必须在明确定义的状态下初始化。对于原子,通常可以通过光学泵(optical pumping)方便地实现这一点。光学泵的基本原理是将原子驱动到驱动不再作用的状态。通常使用圆偏振光将原子泵到极端的塞曼能级之一。目标状态在不到1微秒内被占据的概率大于0.99。初始化保真度通常受到驱动激光沿量子比特优选轴的偏振质量的限制。

在量子算法结束时,量子寄存器需要被测量。通过仅与量子比特的一个能级耦合的辐射可以实现这一点。

\subsubsection{量子门的物理实现}
所有量子算法都可以分解为一系列单量子比特操作加上一个特定的双量子比特操作,例如条件相位门、受控非门或 SWAP 门。

\bold{1.单量子门}

单量子门作用于量子比特上对应于态矢在布洛赫球进行的旋转。
通过两个量子比特能级之间的拉比翻转实现单量子比特操作。数学上,可以利用旋转 $R^C(\theta, \varphi)$ 作用于状态向量 $\alpha |0\rangle + \beta |1\rangle$来描述共振辐射诱导的耦合效应。

\[
R^C(\theta, \varphi) = \exp \left( i \frac{\theta}{2} \left( e^{i \varphi} \sigma_+ + e^{-i \varphi} \sigma_- \right) \right) = I \cos \frac{\theta}{2} + i (\sigma_x \cos \varphi - \sigma_y \sin \varphi) \sin \frac{\theta}{2}
\]
\[
= \begin{pmatrix} \cos \frac{\theta}{2} & i e^{i \varphi} \sin \frac{\theta}{2} \\ i e^{-i \varphi} \sin \frac{\theta}{2} & \cos \frac{\theta}{2} \end{pmatrix},
\]

其中 $\sigma_+ = \begin{pmatrix} 0 & 0 \\ 0 & 0 \end{pmatrix}$ 和 $\sigma_- = \begin{pmatrix} 0 & 0 \\ 0 & 0 \end{pmatrix}$ 是相应的原子激发和湮灭算符。$\sigma_x = \begin{pmatrix} 0 & 0 \\ 0 & 0 \end{pmatrix}$ 和 $\sigma_y = \begin{pmatrix} 0 & -i \\ i & 0 \end{pmatrix}$ 是泡利自旋矩阵。角度 $\theta$ 和 $\varphi$ 指定旋转。
任何 $\sigma_x$ 和 $\sigma_y$ 操作的线性组合都可以通过激光脉冲实现。绕 z 轴的旋转可以分解为绕 x 轴和 y 轴的旋转。
通过调整激光脉冲,可以在布洛赫球上实现任意旋转,从而操控量子态实现任意的单量子门。

\bold{2.双量子门}

Cirqac-Zoller门:

Cirqac 和 Zoller 提出了在离子串中两个离子之间执行双量子比特门的以下过程:首先,第一个量子比特的量子信息被转移到离子晶体的模式(the bus mode)的运动自由度中。由此产生的运动状态不仅影响被寻址的离子本身,还影响整个离子串。因此,在第二个离子上,可以执行以离子串的运动状态为条件的操作。最后,量子信息被映射回第一个离子的量子比特上。利用Cirqac-Zoller门可以得到普通的受控非门或零受控非操作。


M\o{}lmer-S\o{rensen} 门:

利用双色激光场,通过调整激光频率接近集体模式的红边带和蓝边带,实现量子比特之间的纠缠。

基本原理如下:两个离子都受到双色激光场的照射,频率为 $\omega_0 \pm (\omega_{\text{qubit}} + \delta)$,分别调谐接近集体模式的红边带和蓝边带。这两个频率的总和是量子比特频率 $\omega_{\text{qubit}}$ 的两倍,然而,每个激光场本身并不谐振于任何能级。因此,两个离子只能集体改变它们的状态,并适当选择相互作用时间,实现动力学

\[
\begin{array}{l}
|ee\rangle \rightarrow (|ee\rangle + i|gg\rangle)/\sqrt{2} \\
|eg\rangle \rightarrow (|eg\rangle + i|ge\rangle)/\sqrt{2} \\
|ge\rangle \rightarrow (|ge\rangle + i|eg\rangle)/\sqrt{2} \\
|gg\rangle \rightarrow (|gg\rangle + i|ee\rangle)/\sqrt{2}
\end{array}
\] 

通过引入新基 $\ket{\pm}_i = (\ket{e}_i \pm \ket{g}_i)/\sqrt{2}$,$\ket{\pm}$ 是方程 (15) 描述的酉操作的本征态,并通过门的作用转换为 $\ket{++} \rightarrow \ket{++}, \ket{+-} \rightarrow i\ket{+-}, \ket{-+} \rightarrow i\ket{-+}, \ket{--} \rightarrow \ket{--}$,省略了相位因子 $e^{-i\pi/4}$。


Geometric phase门:

通过激光场对离子施加依赖于电子态的力,使得不同电子态获得不同的相位,从而实现量子门操作。

Geometric phase门也使用两个激光场同时照射多个离子。施加一个依赖于电子态的力,使得对于不同的电子态获得不同的相位。两个激光频率的差异调谐接近于轴频率之一,形成驻波。选择离子之间的距离,使得每个离子在给定时间内经历驻波的相同相位,如果它们具有相同的电子态,则离子被推在同一方向上。如果离子处于不同的电子态,则作用于离子上的力不同,可以激发呼吸模式(breathing mode)。从运动模式的驱动失谐选择使得运动和驱动之间的相位在门时间的一半后改变符号。通过这种方式,离子串在完整的门时间后被驱动回到原始运动状态。这确保在门操作后运动与电子态解耦合。与未激发运动相比,中间能量增加导致所需的相位因子,离子获得一个非线性依赖于两个离子内部状态的相位因子。

\subsubsection{具体应用}

\bold{1.Deutsch-Josza 算法}

Deutsch-Josza 算法用于检测未知函数的奇偶性。对单个比特,存在四种不同的函数将一个 qubit 值 $a = \in \{0, 1\}$ 映射到另一个。这些函数可以分为常(偶数)函数 $\left(f_1(a) = 0\right.$ 和 $f_2(a) = 1$)和平衡(奇数)函数 $\left(f_3(a) = a\right.$ 和 $f_4(a) = NOT a$)。使用经典计算机,需要至少调用 $f_n$ 两次以确定 $f_n$ 是奇数还是偶数,即需要计算 $f_n(0)$ 和 $f_n(1)$。然而,通过量子力学形式化程序,只需调用一次即可确定 $f_n$ 是常数还是平衡数。

为了量子力学地表述问题,函数 $f_n$ 必须推广为接受量子比特作为输入。在量子力学框架内,所有操作都是幺正的,因此添加另一个量子比特(辅助量子比特)以允许非可逆函数 $f_1$ 和 $f_2$。重新表述任务,量子比特 $|a\rangle$ 持有输入变量 $x$,而量子比特 $|w\rangle$(工作量子比特)将接收 $f_n(a)$ 的评估结果加上量子比特 $|w\rangle$ 的初始值 $w$ 以保证可逆性。因此,我们定义 $U_{f_n}$ 表示在 $|w\rangle|a\rangle$ 上作用的函数实现,分别取 $w$ 和 $a$ 的值:

\begin{equation}
U_{f_n} |w\rangle |a\rangle = |f_n(a) \oplus w\rangle |a\rangle. \tag{29}
\end{equation}

这里,$\oplus$ 表示模 2 加法。

DJ-算法包括以下步骤:

\begin{enumerate}
    \item 初始化系统到状态 $\ket{0_a} \ket{1_w}$。
    \item 将输入量子比特 $|a\rangle$ 转移到 $(|0\rangle + |1\rangle)/\sqrt{2}$,工作量子比特 $|w\rangle$ 转移到 $(|0\rangle - |1\rangle)/\sqrt{2}$,使用 Hadamard 操作 $R_y$。
    \item 通过实现 $U_{f_n}$ 调用这些叠加值。
    \item 通过在 $|a\rangle$ 上应用逆 Hadamard 操作 $(R_y)$ 关闭干涉。
    \item 在 $|a\rangle$ 中读取结果。
\end{enumerate}

离子阱实验仅使用单个 $^{40} Ca^{+}$ 离子。内部状态作为量子比特 $|a\rangle$ 来保存函数的输入变量,逻辑赋值为 $|0\rangle \equiv |S\rangle$,而轴振动自由度用作工作量子比特 $|w\rangle$(逻辑赋值为 $|0\rangle \equiv |1\rangle_{ax}$ 和 $|1\rangle \equiv |0\rangle_{ax}$)。因此,基态冷却到 $|S, 0\rangle$ 初始化系统到 $|0_a\rangle|1_w\rangle$,如所需。量子比特在离子运动态中的编码特性是,必须确保系统不离开计算子空间 $\{|S, 0\rangle, |D, 0\rangle, |S, 1\rangle, |D, 1\rangle$。在实验中,通过复合脉冲技术实现。此外,必须在运动自由度上执行单量子比特操作。为此,运动自由度中的量子信息被交换到电子自由度,使得可以使用普通载流子脉冲。最后,量子信息被交换回运动自由度。Hadamard 旋转 $R_{yw}$ 和 $R_{yw}$ 被吸收到函数定义中。

为了确定实现函数的类别,只需测量量子比特 $|a\rangle$。在 $|0\rangle$ 中找到 $|a\rangle$ 表示函数是常数,在 $|1\rangle$ 中找到 $|a\rangle$ 表示函数是平衡函数。


\bold{2.量子模拟}

一个任意势中的自旋-1/2 粒子的量子动力学可以通过单个囚禁离子有效地模拟。 NIST-group模拟了光学 Mach-Zehnder 干涉仪的作用,从而实现了单离子的首次量子模拟。在这项工作中,光束分束以以下方式实现:离子的电子态表示一个输入模式中的光子数,而离子的运动态描述一个输出模式的状态。蓝边带的 $R^+(\pi/2, 0)$ 脉冲可以将光子从一个模式转移到另一个模式,从而实现束分束分束器。因此,蓝边带的两个 $R^+(\pi/2)$-脉冲序列类似于 Mach-Zehnder 干涉仪的两个光束分束器的作用。此外,使用二阶和三阶边带脉冲(失谐 $\Delta = n\omega_t, n = \{2, 3\}$)可以实现非线性干涉仪,其中一个模式中的一个光子可以在另一个模式中生成两个或更多光子。


\subsubsection{面临的挑战}
\begin{paralist}
    \item 量子比特的扩展性:目前离子阱量子计算的规模有限,扩展到更多量子比特是一个重要的挑战。需要解决的问题包括离子链的稳定性、运动模式的复杂性等。
    \item 量子门的保真度:提高量子门操作的保真度是实现容错量子计算的关键。目前,两比特门的保真度仍有限,需要进一步优化激光脉冲和离子的运动控制。
    \item 量子比特的相干时间:延长量子比特的相干时间对于量子计算的可靠性至关重要。通过改进离子阱的设计、降低环境噪声等手段,可以提高量子比特的相干时间。
    \item 量子计算的容错性:实现容错量子计算需要满足一定的错误率阈值。目前,离子阱量子计算的错误率仍高于这一阈值,需要进一步降低错误率并优化量子纠错协议
    \item 量子计算的实用性:尽管离子阱量子计算在实验上取得了一些进展,但要实现实用化的量子计算机仍面临许多挑战,包括提高量子比特的数量、量子门的保真度、量子比特的相干时间等。此外,还需要开发高效的量子算法和量子编程语言,以充分利用量子计算的优势。
\end{paralist}

\subsection{超导量子计算}
与电子自旋、量子点、囚禁离子等其他量子比特平台相比,超导量子位的一个显著特征是它们的能级谱由电路元件参数控制,因此是可配置的;它们可以被设计成表现出具有所需性质的“类原子”能谱。因此,超导量子位也经常被称为“人造原子”,提供了可能的量子位属性和操作机制的丰富参数空间,在跃迁频率、非谐性和复杂性方面具有可预测的性能。
\subsubsection{量子位的物理实现}
首先,从经典线性LC谐振电路的开始。在这个系统中,能量在电容C的电能和电感L的磁能之间振荡。每个元件的瞬时、时间依赖的能量由其电流和电压导出:
\[
E(t) = \int_{-\infty}^{t} V(t')I(t')dt',
\]

其中 \(V(t')\) 和 \(I(t')\) 分别表示电容器或电感器的电压和电流。

利用经典力学中使用的标准方法:拉格朗日-哈密顿公式推导经典哈密顿量。用其广义电路坐标之一,电荷或磁通量,来表示电路元件。磁通量定义为电压的时间积分
\[
\Phi(t) = \int_{-\infty}^{t} V(t')dt'.
\]

利用关系 \(V = L \frac{dI}{dt}\) 和 \(I = C \frac{dV}{dt}\),并进行积分,可以写出电容和电感的能量:
\[
T_C = \frac{1}{2} C \dot{\Phi}^2,
\]
\[
U_L = \frac{1}{2L} \Phi^2.
\]

拉格朗日量定义为动能项和势能项之间的差:
\[
\mathcal{L} = T_C - U_L = \frac{1}{2} C \dot{\Phi}^2 - \frac{1}{2L} \Phi^2.
\]

从拉格朗日量进一步推导出哈密顿量,为此需要计算与磁通量共轭的动量,对应电容器上的电荷
\[
Q = \frac{\partial \mathcal{L}}{\partial \dot{\Phi}} = C \dot{\Phi}.
\]

系统的哈密顿量现在定义为
\[
H = Q \dot{\Phi} - \mathcal{L} = \frac{Q^2}{2C} + \frac{\Phi^2}{2L} \equiv \frac{1}{2} CV^2 + \frac{1}{2} LI^2,
\]

这个哈密顿量类似于具有质量 \(m = C\) 和共振频率 \(\omega = 1/\sqrt{LC}\) 的机械谐振子的哈密顿量,其在位置 \(x\) 和动量 \(p\) 坐标中的形式为 \(H = p^2/2m + m\omega^2x^2/2\)。

为了得到系统的量子力学描述,我们需要将电荷和磁通量坐标变为量子算符,而经典坐标满足泊松括号
\[
\{f,g\} = \frac{\delta f}{\delta \Phi} \frac{\delta g}{\delta Q} - \frac{\delta g}{\delta \Phi} \frac{\delta f}{\delta Q}
\]
\[
\rightarrow \{\Phi, Q\} = \frac{\delta \Phi}{\delta \Phi} \frac{\delta Q}{\delta Q} - \frac{\delta Q}{\delta \Phi} \frac{\delta \Phi}{\delta Q} = 1 - 0 = 1,
\]

量子算符同样满足对易关系:
\[
[\hat{\Phi}, \hat{Q}] = \hat{\Phi}\hat{Q} - \hat{Q}\hat{\Phi} = i\hbar,
\]

为了简化后面将省略算符上的帽子。

在简单的LC谐振电路中,电感 \(L\) 和电容 \(C\) 都是线性电路元件。定义归一化磁通量 \(\phi \equiv 2\pi \Phi/\Phi_0\) 和归一化电荷 \(n = Q/2e\),我们可以写出电路的量子力学哈密顿量
\[
H = 4E_C n^2 + \frac{1}{2} E_L \phi^2,
\]
其中 \(E_C = e^2/(2C)\) 是将“每个”库珀对电子添加到容器上所需的充电能量,\(E_L = (\Phi_0/2\pi)^2/L\) 是感应能量,其中 \(\Phi_0 = h/(2e)\) 是超导磁通量子。此外,量子算符 \(n\) 是电路上多余的库珀对数,而 \(\phi\) ——归一化磁通量——被称为电感器上的“规范不变相位”。这两个算符形成一个规范共轭对,满足对易关系 \([\phi, n] = i\)。

哈密顿量与描述一维二次势中的粒子的哈密顿量相同,即量子谐振子(QHO)。我们可以将 \(\phi\) 视为广义位置坐标,因此第一项是动能,第二项是势能。令$\omega_r = \sqrt{8 E_L E_C}/\hbar = 1/\sqrt{LC}$可以得到:
\[
H = \hbar \omega_r \left( a^\dagger a + \frac{1}{2} \right),
\]

其中 $a^\dagger (a)$ 是谐振子的单个激发的产生(湮灭)算符。

用约瑟夫森结替换QHO中的线性电感L,扮演非线性电感器的角色。
约瑟夫森结的势能可以从约瑟夫森关系导出
\[
I = I_c \sin(\phi), \quad V = \frac{\hbar}{2e} \frac{d\phi}{dt},
\]

导致修改后的哈密顿量
\[
H = 4E_C n^2 - E_J \cos(\phi),
\]

其中 $E_C = e^2/(2C_\Sigma)$,$C_\Sigma = C_s + C_J$ 是总电容,包括旁路电容 $C_s$ 和结的自电容 $C_J$,$E_J = I_c \Phi_0/2\pi$ 是约瑟夫森能量,$I_c$ 是结的临界电流。在引入约瑟夫森结到电路后,势能具有余弦形式,使得能谱非简并。从而获得一个唯一可寻址的量子两能级系统。

通过组合电容、电感和约瑟夫森、调整电路参数,可以设计量子比特的能级结构,使其具有所需的量子比特特性。
\subsubsection{量子门的物理实现}
绕布洛赫球上的任意轴旋转的能力,加上任何纠缠的2量子位操作,即能实现任意的量子门。

\bold{1.单量子门}

基于微波驱动的单比特旋转。
通过将超导量子比特与微波驱动线(drive-line)耦合,利用电容耦合实现微波脉冲对量子比特的控制。
在旋转波近似下,微波脉冲可以实现绕x轴或y轴的旋转操作。例如,与量子比特共振的脉冲可以实现$\pi$脉冲,将量子比特从基态翻转到激发态。

将量子比特建模为一个谐振子,其经典哈密顿量如下:
\[
H = \frac{\tilde{Q}(t)^2}{2C_\Sigma} + \frac{\Phi^2}{2L} + \frac{C_d}{C_\Sigma} V_d(t) \tilde{Q}, \tag{74}
\]

其中 $C_\Sigma = C + C_d$ 是总电容,$\tilde{Q} = C_\Sigma \Phi - C_d V_d(t)$ 是电路的归一化电荷变量。将磁通量和电荷变为量子算符,并假设与驱动线的弱耦合,使得 $\tilde{Q} \approx \hat{Q}$,从而得到
\[
H = H_{LC} + \frac{C_d}{C_\Sigma} V_d(t) \hat{Q}, \tag{75}
\]

其中 $H_{LC} = \hat{Q}^2/(2C) + \hat{\Phi}^2/(2L)$,保留了与动态变量耦合的项。类似于 $(x, p)$ 空间中谐振子的动量算符,用升降算符表示电荷
\[
\hat{Q} = -i Q_{zpf} (a - a^\dagger), \tag{76}
\]

其中 $Q_{zpf} = \sqrt{\hbar/2Z}$ 是零点电荷涨落,$Z = \sqrt{L/C}$ 是电路对地的阻抗。因此,与驱动线电容耦合的LC振子可以写为

\[
H = \omega \left(a^\dagger a + \frac{1}{2}\right) - \frac{C_d}{C_\Sigma} V_d(t) i Q_{zpf} (a - a^\dagger). \tag{777}
\]

最后,通过截断开到振子的最低跃迁,我们可以在整个过程中用 $\sigma^-$ 替换 $a$,用 $\sigma^+$ 替换 $a^\dagger$,得到
\[
H = \underbrace{-\frac{\omega_q}{2} \sigma_z}_{H_0} + \underbrace{\Omega V_d(t) \sigma_y}_{H_d}, \tag{78}
\]

其中 $\Omega = (C_d/C_\Sigma) Q_{zpf}$ 且 $\omega_q = (E_1 - E_0)/\hbar$。

选择一个与量子比特频率 $\omega_q$旋转的参考系。考虑状态 $|\psi_0\rangle = (1 1)^T/\sqrt{2}$。根据时间依赖的薛定谔方程,这个状态按照

\[
|\psi_0(t)\rangle = U_{H_0}|\psi_0\rangle = \frac{1}{\sqrt{2}} \left( e^{i\omega_q t/2} \right)
\]

其中 $U_{H_0}$ 是对应于 $H_0$ 的传播子。通过计算,例如,$\langle \psi_0|\sigma_x|\psi_0\rangle = \cos(\omega_q t)$,很明显,由于 $\sigma_z$ 项,相位以 $\omega_q$ 的频率旋转。为此,我们定义 $U_{rf} = e^{iH_0t} = U_t^{\dagger}$,旋转参考系中的新状态是 $|\psi_{rf}(t)\rangle = U_{rf}|\psi_0\rangle$。在这个新参考系中的时间演化再次从薛定谔方程中找到( $\dot{Q} = \partial/\partial t$)

\[
i\partial_t |\psi_{rf}(t)\rangle = i(\partial_t U_{rf})|\psi_0\rangle + iU_{rf}(\partial_t|\psi_0\rangle)
\]
\[
= iU_{rf} U_t^{\dagger}|\psi_{rf}\rangle + U_{rf}H_0|\psi_0\rangle,
\]
\[
=(iU_{rf} U_t^{\dagger} + U_{rf}H_0U_{rf}^{\dagger})|\psi_{rf}\rangle.
\]

可以将方程中的括号内的项视为旋转参考系中 $H_0$ 的形式。简单的插入表明 $H_0 = 0$,这是预期的(旋转参考系应该处理时间依赖性)。

旋转参考系中 $H_d$ 的形式为
\[
H_d = \Omega V_d(t)(\cos(\omega_q t)\sigma_y - \sin(\omega_q t)\sigma_x)
\]

假设电压的时间依赖部分 $\left(V_d(t) = V_0 v(t)\right)$ 具有通用形式
\[
v(t) = s(t) \sin(\omega_d t + \phi)
\]
\[
= s(t) \left( \cos(\phi) \sin(\omega_d t) + \sin(\phi) \cos(\omega_d t) \right)
\]

其中 $s(t)$ 是无量纲包络函数,因此驱动的幅度由 $V_0 s(t)$ 设置。采用定义
\[
I = \cos(\phi) \text{(“同相”分量)}
\]
\[
Q = \sin(\phi) \text{(“反相”分量)}
\]

旋转参考系中的驱动哈密顿量形式为

\[
\tilde{H}_d =  \Omega V_0 s(t) \left( I \sin(\omega_d t) - Q \cos(\omega_d t) \right) \times \left( \cos(\omega_q t) \sigma_y - \sin(\omega_q t) \sigma_x \right)
\]

丢弃平均为零的快速旋转项(即,具有 $\omega_q + \omega_d$ 的项),称为旋波近似(RWA),得到

\[
\tilde{H}_d =  \frac{1}{2} \Omega V_0 s(t) \left[ \left( -I \cos(\delta \omega t) + Q \sin(\delta \omega t) \right) \sigma_x  + \left( I \sin(\delta \omega t) - Q \cos(\delta \omega t) \right) \sigma_y \right]
\]

其中 $\delta \omega = \omega_q - \omega_d$。使用RWA在旋转参考系中写出驱动哈密顿量

\[
\tilde{H}_d = -\frac{\Omega}{2} V_0 s(t) \left( \begin{array}{cc} 0 & e^{i(\delta \omega t + \phi)} \\ e^{-i(\delta \omega t + \phi)} & 0 \end{array} \right)
\]

假设在量子比特频率处应用一个脉冲,使得 $\delta \omega = 0$,则
\[
\tilde{H}_d = -\frac{\Omega}{2} V_0 s(t) \left( I \sigma_x + Q \sigma_y \right)
\]

表明“同相”脉冲($\phi = 0$,即$I$分量)对应于绕x轴的旋转,而反相脉冲($\phi = \pi/2$,即$Q$分量)对应于绕y轴的旋转。作为一个同相脉冲的具体例子,写出单位算符得到
\[
U_{rf, d}^{\phi=0}(t) = \exp \left( \left[ \frac{i}{2} \Omega V_0 \int_0^t s(t') dt' \right] \sigma_x \right), \tag{92}
\]

这仅依赖于电路的宏观设计参数以及基带脉冲 $s(t)$ 和幅度 $V_0$ 的包络,两者都可以使用任意波形发生器(AWGs)进行控制。定义

\[
\Theta(t) = -\Omega V_0 \int_0^t s(t') dt', \tag{93}
\]

是状态旋转的角度,给定电容耦合、电路的阻抗、幅度 $V_0$ 和波形包络 $s(t)$。要在x轴上实现 $\pi$ 脉冲,需要解方程 $\Theta(t) = \pi$ 并输出与量子比特驱动同相的信号。一系列脉冲 $\Theta_k, \Theta_{k-1}, \ldots, \Theta_0$ 转换为一系列作用于量子比特的门操作
\[
U_k \cdots U_1 U_0 = \mathcal{T} \prod_{n=0}^k e^{\left[-\frac{i}{2} \Theta_n(t)(I_n \sigma_x + Q_n \sigma_y)\right]}, \tag{94}
\]

其中 $\mathcal{T}$ 是一个算符,确保脉冲按时间顺序生成,对应于 $U_k \cdots U_1 U_0$。

\bold{2.双量子门}

超导量子计算中部分可实现的双量子门及实现方式如下:
\begin{paralist}
    \item iSWAP门:通过电容耦合两个可调谐频率的量子比特,当两个量子比特共振时,它们之间会发生激发态的交换。将一个量子比特的频率调至与另一个量子比特共振,保持一段时间后,两个量子比特的激发态会相互交换。iSWAP门可以用于生成贝尔态和多量子比特纠缠态,需要多次操作实现CNOT门。
    \item CPHASE门:利用量子比特的高能级与计算子空间之间的相互作用,通过绝热操作实现条件相位门。
    方法:通过调整量子比特的频率,使其接近高能级的共振点,从而在计算子空间中引入条件相位。通过调整量子比特的频率,使其接近高能级的共振点,从而在计算子空间中引入条件相位。使用优化的脉冲序列(如Slepian波形)实现快速且高保真的CPHASE门,适用于量子纠错和量子模拟。
    \item CR门(Cross-Resonance gate):通过微波脉冲驱动一个量子比特,使其频率与另一个量子比特共振,从而实现两比特的相互作用。利用量子比特之间的微波驱动相互作用,实现条件旋转操作。 
\end{paralist}

\subsubsection{面临的挑战}
\begin{paralist}
    \item 噪声与退相干问题:
    超导量子比特的量子态极易受到环境噪声的干扰,导致退相干时间较短,影响量子计算的稳定性和准确性。尽管通过量子纠错技术和优化硬件设计可以降低噪声和退相干的影响,但这些技术的实现仍面临复杂性和资源消耗的挑战。
    \item 可扩展性问题:
    的主流平面结构限制了比特之间的连接性,仅能实现近邻耦合,这增加了运行量子算法时的额外开销。此外,量子比特的控制线数量随着比特数线性增加,布线密度和串扰问题在更高集成度下尤为突出。三维集成技术虽然可以缓解部分问题,但在更高集成度下的布线和串扰抑制仍极具挑战性
    \item 成本与资源限制:
    超导量子计算的研发和应用需要大量的资金和资源投入,包括低温制冷系统、高精度测控设备等。这些高昂的成本限制了技术的进一步发展。
\end{paralist}


