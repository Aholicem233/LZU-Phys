\section{Qiskit的安装及基本使用方法}
\label{附录A}
\subsection{Qiskit的安装}
\textbf{Qiskit的安装要求:}

Python版本:建议使用Python 3.6及以上版本。这是Qiskit运行的基础环境,确保Python版本符合要求可以避免安装过程中出现兼容性问题。

操作系统:Qiskit支持64位操作系统,如Ubuntu 16.04、macOS 10.12.6、Windows 7及以上版本。

基础函数库:Qiskit安装过程中可能需要一些基础函数库,如Numpy等。

\subsubsection{下载并安装Anaconda}
\begin{paralist}
    \item 下载Anaconda:进入官网下载(https://www.anaconda.com/download)

    \item 安装conda 
\end{paralist}
\subsubsection{创建虚拟环境并安装Qiskit}
\begin{paralist}
    \item 虚拟环境的创建:打开Anaconda Prompt输入并运行\textsf{conda create -n py python=3.8}
    \item 激活虚拟环境:输入并运行\textsf{conda activate py}
    \item 安装Qiskit:在虚拟环境下输入并运行\textsf{pip install qiskit[visualization]}(遇到网络问题可以配置使用镜像源)
\end{paralist}
\subsubsection{Jupyter Notebook使用}
在创建好的虚拟环境中输入:
\begin{paralist}
    \item \textsf{conda install ipykernel}
    \item \textsf{ipython kernel install --user --name=pykernel}
\end{paralist}

在Anaconda安装目录下打开Jupyter Notebook,点击New且选择pykernel创建ipynb文件,即可开始编写、运行程序。
\figu{0.4}{figures/ju.png}
\subsection{Qiskit的基本使用方法}
Qiskit能实现多个功能。下面主要介绍如何创建量子线路并绘制、如何模拟器运行量子线路。

\bold{1.创建、绘制量子线路}

Qiskit 创建与绘制量子线路的主要步骤如下:

\begin{py}
\begin{lstlisting}
#1.导入库
from qiskit import(QuantumCircuit,transpile)
from qiskit_aer import Aer
from qiskit.visualization import plot_histogram
#2.创建量子线路
circuit = QuantumCircuit(2,2)#创建两个经典位与量子位
circuit.h(0)#在第0个量子位上添加H门
circuit.cx(0,1)#添加受控非门,控制位为0,目标位为1
circuit.measure([0,1],[0,1])
#(qubit, cbit)将qubit测量结果存储到cbit中
#3.绘制量子线路
circuit.draw(output='mpl',filename='qc.png')#获得png图片
\end{lstlisting}
\end{py}
\fig{输出的量子线路图}{0.6}{code/qc.png}

\bold{2.模拟器运行}

QasmSimulator和StatevectorSimulator为Qiskit中两类常用的模拟器,它们的特点如下:
\begin{paralist}
    \item QasmSimulator:模拟量子线路的多次运行并返回测量结果的概率分布。
    \item StatevectorSimulator:返回量子线路的最终状态向量。主要用于分析量子线路的量子态,不涉及测量操作。
\end{paralist}

QasmSimulator使用示例:
\begin{py}
\begin{lstlisting}
#QasmSimulator使用示例
from qiskit import QuantumCircuit, transpile
from qiskit_aer import Aer
from qiskit.visualization import plot_histogram

# 创建量子线路
qc = QuantumCircuit(2, 2)
qc.h(0)
qc.cx(0, 1)
qc.measure([0, 1], [0, 1])

# 选择QasmSimulator
simulator = = Aer.get_backend('qasm_simulator')

# 编译电路
transpiled_circuit = transpile(qc, simulator)

# 运行电路并获取结果
job = simulator.run(transpiled_circuit, shots=1024)
result = job.result()

# 获取测量结果
counts = result.get_counts(qc)
print("实验结果:", counts)

# 可视化
out = plot_histogram(counts)
out.savefig('out.png')
\end{lstlisting}
\end{py}

\fig{QasmSimulator使用示例输出结果}{0.6}{code/out.png}
输出结果与\( |\Phi^{+}\rangle = \frac{1}{\sqrt{2}} (|00\rangle + |11\rangle) \),$|00\rangle$与$|11\rangle$出现概率为$50\%$对应。

StatevectorSimulator使用示例:

\begin{py}
\begin{lstlisting}
#StatevectorSimulator使用示例
from qiskit import QuantumCircuit, transpile
from qiskit_aer import Aer
from qiskit.visualization import plot_state_city

# 创建量子线路
qc = QuantumCircuit(2)
qc.h(0)
qc.cx(0, 1)

# 选择StatevectorSimulator
simulator = Aer.get_backend('statevector_simulator') 


# 编译电路
transpiled_circuit = transpile(qc, simulator)

# 运行电路并获取结果
job = simulator.run(transpiled_circuit)
result = job.result()

# 获取状态向量
statevector = result.get_statevector(qc)

#可视化
out=plot_state_city(statevector)
out.savefig('out1.png')
\end{lstlisting}
\end{py}
\fig{StatevectorSimulator使用示例输出结果}{0.8}{code/out1.png}
输出结果与 \( |\Phi^{+}\rangle = \frac{1}{\sqrt{2}} (|00\rangle + |11\rangle) \) 的密度矩阵一致,即

\[
\rho_{\Phi^{+}} = |\Phi^{+}\rangle \langle \Phi^{+}|
= \begin{bmatrix} \frac{1}{\sqrt{2}} \\ 0 \\ 0 \\ \frac{1}{\sqrt{2}} \end{bmatrix} \begin{bmatrix} \frac{1}{\sqrt{2}} & 0 & 0 & \frac{1}{\sqrt{2}} \end{bmatrix}
= \frac{1}{2} \begin{bmatrix} 1 & 0 & 0 & 1 \\ 0 & 0 & 0 & 0 \\ 0 & 0 & 0 & 0 \\ 1 & 0 & 0 & 1 \end{bmatrix}
\]

除以上两个模拟器外,Qiskit中还有模拟器Uintary Simulator。通过UnitarySimulator,可以方便地得到整个量子线路对应的单位矩阵,对于理论分析和验证量子线路的正确性非常有用。使用方法与上面类似。
\begin{py}
\begin{lstlisting}
#UnitarySimulator使用示例
from qiskit import QuantumCircuit, transpile
from qiskit_aer import Aer

# 创建量子线路
qc = QuantumCircuit(2)
qc.h(0)
qc.cx(0, 1)

# 选择 unitary_simulator
simulator = Aer.get_backend('unitary_simulator')

# 编译电路
transpiled_circuit = transpile(qc, simulator)

# 运行电路并获取结果
job = simulator.run(transpiled_circuit)
result = job.result()

# 获取单位矩阵
unitary = result.get_unitary(qc)
print("单位矩阵:\n", unitary)  
\end{lstlisting}
\end{py}
\fig{UnitarySimulator使用示例输出结果}{0.8}{figures/uni.png}